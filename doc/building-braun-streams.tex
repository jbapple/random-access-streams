\documentclass{llncs}
%\usepackage{amsthm}
\usepackage{amssymb,amsmath,algorithmic,algorithm,verbatim,alltt}

\DeclareMathOperator{\oddFrom}{oddFromEven}
\DeclareMathOperator{\nil}{\langle\rangle}
%\DeclareMathOperator{\cons}{cons}
\DeclareMathOperator{\true}{true}
\DeclareMathOperator{\false}{false}
\DeclareMathOperator{\head}{head}
\DeclareMathOperator{\evens}{evens}
\DeclareMathOperator{\odds}{odds}
\DeclareMathOperator{\fmap}{fmap}
\DeclareMathOperator{\iter}{iterate}
\newcommand{\ord}[1]{\|#1\|}
\newcommand{\cons}[2]{#1:#2}
\DeclareMathOperator{\ev}{ev}
\DeclareMathOperator{\od}{od}
%\renewcommand{\algorithmiccomment}[1]{// #1}

\begin{comment}
\newtheorem{theorem}{Theorem}
\newtheorem{claim}[theorem]{Claim}
\newtheorem{lemma}[theorem]{Lemma}
\newtheorem{fact}[theorem]{Fact}
\newtheorem{corollary}[theorem]{Corollary}
\newtheorem{observation}[theorem]{Observation}
\newtheorem{proposition}[theorem]{Proposition}

%\theoremstyle{definition}
\newtheorem{definition}{Definition}

\end{comment}

\begin{document}

\title{Building Braun streams efficiently}

\author{Jim Apple}

\institute{}

\maketitle

\section*{Abstract}

Braun trees are a functional data structure with logarithmic-time indexing, update, and deque operations.\cite{hoogerwoord}
They are mostly superseded by finger trees, which decrease to a constant the time required for deque operations.\cite{kaplan96purely,HinzePat}
The restrictive structural constraints that give Braun trees slower deque operations than finger trees make them well-suited for operations on infinite lists, or {\em streams}.

This paper presents two new algorithms for building Braun streams from cons lists.
The first algorithm builds a Braun stream from a cons stream; previous algorithms to build Braun trees either require superlinear time or terminate only for finite cons lists.\cite{okasakiBraun} 
The second algorithm builds a cyclic Braun stream from a possibly finite list; when the list is finite and of length $n$, it builds a Braun stream that uses no more than $O(n^2)$ space, even when forced to arbitrary depth.
In contrast, circular streams represented with finger trees use at least $O(i)$ space when forced to depth $i$.

Correctness proofs of the two algorithms are presented along with formal verifications using the Coq proof assistant.

\section{Introduction}

To achieve these constant-time bounds, however, skew lists and finger trees relax their structural constraints. 
This prevents some stream constructions from terminating. 
For instance, \verb|cons| pattern matches on its second argument to determine its top-level constructor, so \verb|let ones = cons 1 ones| can never produce a top-level constructor.

In addition, the complexity of the structural constraints on skew lists and finger trees increase the memory usage of certain streams.
For instance, the stream \verb|ones| mentioned above can be stored in a Braun stream such that arbitrary indices may be retrieved without allocating more than a constant amount of memory.
Finger trees and skew lists, on the other hand, will both allocate $O(\log i)$ memory when the $i$th element is retrieved.

What does linear time mean on infinite streams?

\section{Braun streams}

Braun streams are simply infinite full binary trees.
The $0$th element is stored in the root of the tree, while odd-indexed elements appear in the left subtree and even-indexed elements appear in the right.
The subtrees of a Braun tree are also Braun trees, after reindexing by order.
That is, the 0th entry of any subtree is at its root, and its 1st, 3rd, 5th, ... elements are in its left subtree, etc.

To make this recursive relation clear, we index Braun streams by finite lists of booleans.
The root element is indexed by the empty list.
The odd-indexed elements are indexed by lists that start with \verb|True|, the even-indexed elements by lists that start with \verb|False|.
There is an obvious bijection between finite lists of booleans and integers;
it is expressed in \verb|ord| and \verb|list| in Figure~\ref{basicCode}.

\begin{figure}
\begin{alltt}
data Stream a = Stream a (Stream a) (Stream a) 

bat :: Stream a -> [Bool] -> a
bat (Stream h _ _) [] = h
bat (Stream _ o _) (True:r) = bat o r
bat (Stream _ _ e) (False:r) = bat e r

ord :: [Bool] -> Integer
ord [] = 0
ord (True:r) = 1 + 2*(ord r)
ord (False:r) = 2 + 2*(ord r)

list :: Integer -> [Bool]
list 0 = []
list n = 
    case x `mod` 2 of
      | 1 -> False:(list ((x-1)`div`2))
      | 0 -> True:(list ((x-2)`div`2))

head :: Stream a -> a
head (Stream h _ _) = h

odds :: Stream a -> Stream a
odds (Stream _ o _) = o

evens :: Stream a -> Stream a
evens (Stream _ _ e) = e
\end{alltt}
\caption{The basic definitions of Braun streams with their ordering as written in Haskell.}
\label{basicCode}
\end{figure}

\section{Building Braun streams}

The algorithm for building a Braun stream from a linear stream (along with supporting code) is displayed in Figure~\ref{iterateCode}.

\begin{figure}
\begin{alltt}
instance Functor Stream where
    fmap f (Stream p q r) = Stream (f p) (fmap f q) (fmap f r)

oddFromEven :: (a -> b) -> b -> Stream a -> Stream b
oddFromEven f x  \(\sim\)(Stream h od ev) =
    Stream x (oddFromEven f (f h) ev) (fmap f od)

iterate :: (a -> a) -> a -> Stream a
iterate f x =
    let ev = fmap f od
        od = oddFromEven f (f x) ev
    in
      Stream x od ev

data LStream a = Cons a (LStream a)

ltail :: LStream a -> LStream a
ltail (Cons _ xs) = xs

lhead :: LStream a -> a
lhead (Cons x _) = x

fromLStream :: LStream a -> Stream a
fromLStream s = fmap lhead (iterate ltail s)
\end{alltt}
\caption{The algorithm for building a Braun stream from a linear stream in Haskell}
\label{iterateCode}
\end{figure}

For the proofs, \verb|ord l| is written $\ord{l}$ and \verb|bat s l| is written $s@l$.

\begin{lemma}\label{oddFromLemma}
For any $f, b, x, e, k$
if
\begin{displaymath}
\forall j . \ord{j} < \ord{b} \implies e@j = f^{2^k\ord{j}+2^{k+1}-2}x
\end{displaymath}
then
\begin{displaymath}
(\oddFrom f\ (f^{2^k-1}x)\ e)@b = f^{2^k\ord{b}+2^k-1}x
\end{displaymath}
\end{lemma}
\begin{proof}
We prove this by structural induction on $b$. If $b = \nil$, then 
\begin{displaymath}
\begin{array}{rcl}
(\oddFrom f\ (f^{2^k-1}x)\ e)@b & = & (\oddFrom f\ (f^{2^k-1}x)\ e)@\nil \\
& = & f^{2^k-1}x \\
& = & f^{0+2^k-1}x \\
& = & f^{2^k\ord{\nil}+2^k-1}x \\
& = & f^{2^k\ord{b}+2^k-1}x
\end{array}
\end{displaymath}

If $b = \cons{\true}{d}$,
\begin{displaymath}
\begin{array}{rcl}
(\oddFrom f\ (f^{2^k-1}x)\ e)@(\cons{\true}{d})& = & \\
(\oddFrom f\ (f (\head e))\ (\evens e))@d& = & \\
(\oddFrom f\ (f (f^{2^{k+1}-2}e))\ (\evens e))@d& = & \\
(\oddFrom f\ (f^{2^{k+1}-1}e)\ (\evens e))@d&  & 
\end{array}
\end{displaymath}

Now, for all $g$ such that $\ord{g} < \ord{d}$, $\ord{g} \leq \ord{d}-1$, so 
\begin{displaymath}
\ord{\cons{\false}{g}} = 2 + 2\ord{g} \leq 2 + 2(\ord{d} -1) = 2\ord{d} < 1+2\ord{d} = \ord{\cons{\true}{d}}
\end{displaymath}

So, $\ord{\cons{\false}{g}} < \ord{b}$, and 
\begin{displaymath}
\begin{array}{rcccl}
(\evens e)@g & = & e@(\cons{\false}{g}) & = & f^{2^k\ord{\cons{\false}{g}}+2^{k+1}-2}x \\
%& & & = & f^{2^k(2+2\ord{g})+2^{k+1}-2}x \\
%& & & = & f^{2^{k+1} + 2^{k+1}\ord{g}+2^{k+1}-2}x \\
& & & = & f^{2^{k+1}\ord{g}+2^{k+2}-2}x 
\end{array}
\end{displaymath}

Since this holds for all $g < d$ we can invoke the induction hypothesis with $k := k+1$ and $e := \evens e$.
This implies that

\begin{displaymath}
\begin{array}{rcl}
(\oddFrom f\ (f^{2^k-1}x)\ e)@b & = & (\oddFrom f\ (f^{2^{k+1}-1}x)\ (\evens e))@d \\
& = & f^{2^{k+1}d+2^{k+1}-1} x \\
%& = & f^{2^k2d+2^k+2^k-1} x \\
%& = & f^{2^k(2d+1)+2^k-1} x \\
& = & f^{2^k\ord{\cons{\true}{d}}+2^k-1} x \\
& = & f^{2^k\ord{b}+2^k-1} x
\end{array}
\end{displaymath}

If $b = \cons{\false}{d}$,

\begin{displaymath}
\begin{array}{rcl}
(\oddFrom f\ (f^{2^k-1}x)\ e)@(\cons{\false}{d})& = & (\fmap f\ (\odds e))@d \\
& = & f ((\odds e)@d)
\end{array}
\end{displaymath}

Since
$\ord{\cons{\true}{d}} = 1+2\ord{d} < 2+2\ord{d} = \ord{\cons{\false}{d}}$,

\begin{displaymath}
\begin{array}{rcccl}
f((\odds e)@d) & = & f(e@(\cons{\true}{d})) & = & f(f^{2^k\ord{\cons{\true}{d}}+2^{k+1}-2}x) \\
%& & & = & f^{1+2^k(1+2\ord{d})+2^{k+1}-2}x \\
%& & & = & f^{2^k+2^{k+1}\ord{d}+2^{k+1}-1}x \\
%& & & = & f^{2^k(2+2\ord{d})+2^k-1}x \\
& & & = & f^{2^k\ord{\cons{\false}{d}}+2^k-1}x \\
& & & = & f^{2^k\ord{b}+2^k-1}x
\end{array}
\end{displaymath}

\end{proof}

The full strength of Lemma~\ref{oddFromLemma} is only really needed in the induction hypothesis.
The statement we need for the main proof of \verb|iterate| fixes $k$ at $1$:

\begin{corollary}\label{oddFromCorollary}
For any $f, b, x, e$
if
\begin{displaymath}
\forall j . \ord{j} < \ord{b} \implies e@j = f^{\ord{\cons{\false}{j}}}x
\end{displaymath}
then
\begin{displaymath}
(\oddFrom f\ (f x)\ e)@b = f^{\ord{\cons{\true}{b}}}x
\end{displaymath}
\end{corollary}
\qed

\begin{theorem}\label{iterateCorrect}
\begin{displaymath}
\forall f\ x\ b, (\iter f\ x)@b = f^{\ord{b}} x
\end{displaymath}
\end{theorem}
\begin{proof}

We proceed by well-founded induction on $\ord{b}$.
If $b = \nil$, the equality is trivial.
Otherwise, let $b = \cons{a}{d}$.
There is a lemma that is needed whether $a$ is $\true$ or $\false$

\begin{lemma}\label{iterateSublemma}
\begin{displaymath}
(\oddFrom f\ (f x)\ \ev)@d = f^{2\ord{d}+1}x
\end{displaymath}
\end{lemma}
\begin{proof}
This is a direct result of the corollary as long as 
\begin{displaymath}
\forall j, \ord{j} < \ord{d} \implies \ev @j = f^{2\ord{j}+2}
\end{displaymath}

Since $\ord{j} < \ord{d}$, $2\ord{j}+2 < 2\ord{d}+2 \leq \ord{b}$, we can apply the induction hypothesis to $2\ord{j}+2 = \ord{\cons{\false}{j}}$:

\begin{displaymath}
\begin{array}{rcl}
(\iter f\ x)@(\cons{\false}{j}) & = & f^{\ord{\cons{\false}{j}}} x \\
\ev @j & = & f^{2\ord{j}+2}
\end{array}
\end{displaymath}

\end{proof}

Now, when $a = \true$, $(\iter f\ x)@(\cons{\true}{d}) = (\oddFrom f\ (f x)\ \ev)@d$.
By lemma~\ref{iterateSublemma}, this is just $f^{2\ord{d}+1}x = f^{\ord{b}}x$

When $a = \false$, 

\begin{displaymath}
\begin{array}{rcl}
(\iter f\ x)@(\cons{\false}{d}) & = & (\fmap f\ \od)@d \\
& = & f(\od @d) \\
& = & f((\oddFrom f\ (f x)\ \ev)@d) \\
& = & f(f^{2\ord{d}+1}x) \\
\end{array}
\end{displaymath}

With the last step justified by lemma~\ref{iterateSublemma}. This is just $f^{2\ord{d}+2}x = f^{\ord{\cons{\false}{d}}}x = f^{\ord{b}}x$.

\end{proof}

\section{Formalizing the proof of correctness}

how many lines?

how many generated lines?

hiccups:
* subtraction
* repeated function application not built-in (show equal to another formulation)
* proving termination separately

\section{Conclusions}

Is there a data structure that provides terminating constant-time \verb|cons|?

Also good for Braun trees, because can get index i in time $i$, not $n+i$

Formalizing in a language with extraction

Formalizing time and space complexity

\bibliographystyle{plain}
\bibliography{braun}

\end{document}

% LocalWords:  Braun deque Coq subtree subtrees bijection superlinear
% LocalWords:  oddFromEven
