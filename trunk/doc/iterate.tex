\documentclass{article}
\usepackage{amsthm}
\usepackage{amssymb,amsmath,algorithmic,algorithm,verbatim,alltt}

\DeclareMathOperator{\oddFrom}{oddFromEven}
\DeclareMathOperator{\nil}{\langle\rangle}
\DeclareMathOperator{\cons}{cons}
\DeclareMathOperator{\true}{true}
\DeclareMathOperator{\false}{false}
\DeclareMathOperator{\head}{head}
\DeclareMathOperator{\evens}{evens}
\DeclareMathOperator{\odds}{odds}
\DeclareMathOperator{\fmap}{fmap}
\DeclareMathOperator{\iter}{iterate}
\newcommand{\ord}[1]{\|#1\|}
\DeclareMathOperator{\ev}{ev}
\DeclareMathOperator{\od}{od}
%\renewcommand{\algorithmiccomment}[1]{// #1}

%\begin{comment}
\newtheorem{theorem}{Theorem}
\newtheorem{claim}[theorem]{Claim}
\newtheorem{lemma}[theorem]{Lemma}
\newtheorem{fact}[theorem]{Fact}
\newtheorem{corollary}[theorem]{Corollary}
\newtheorem{observation}[theorem]{Observation}
\newtheorem{proposition}[theorem]{Proposition}

%\theoremstyle{definition}
\newtheorem{definition}{Definition}

%\end{comment}

\begin{document}

\title{A linear-time iterate for Braun streams and its verified proof of correctness}

\author{Jim Apple}

\maketitle

\section*{Abstract}

\cite{HinzePat}

\section{The algorithm}

\begin{alltt}
data Stream a = Stream a (Stream a) (Stream a) 

instance Functor Stream where
    fmap f (Stream p q r) = Stream (f p) (fmap f q) (fmap f r)

oddFromEven :: (a -> b) -> b -> Stream a -> Stream b
oddFromEven f x  \(\sim\)(Stream h od ev) =
    Stream x (oddFromEven f (f h) ev) (fmap f od)

iterate :: (a -> a) -> a -> Stream a
iterate f x =
    let ev = fmap f od
        od = oddFromEven f (f x) ev
    in
      Stream x od ev

bat :: Stream a -> [Bool] -> a
bat (Stream h _ _) [] = h
bat (Stream _ o _) (True:r) = bat o r
bat (Stream _ _ e) (False:r) = bat e r

ord :: [Bool] -> Integer
ord [] = 0
ord (True:r) = 1 + 2*(ord r)
ord (False:r) = 2 + 2*(ord r)

head :: Stream a -> a
head (Stream h _ _) = h

odds :: Stream a -> Stream a
odds (Stream _ o _) = o

evens :: Stream a -> Stream a
evens (Stream _ _ e) = e
\end{alltt}

\section{The proof}

\begin{lemma}\label{oddFromLemma}
For any $f, b, x, e, k$
if
\begin{displaymath}
\forall j . \ord{j} < \ord{b} \implies e@j = f^{2^k\ord{j}+2^{k+1}-2}x
\end{displaymath}
then
\begin{displaymath}
(\oddFrom f\ (f^{2^k-1}x)\ e)@b = f^{2^k\ord{b}+2^k-1}x
\end{displaymath}
\end{lemma}
\begin{proof}
We prove this by structural induction on $b$. If $b = \nil$, then 
\begin{displaymath}
\begin{array}{rcl}
(\oddFrom f\ (f^{2^k-1}x)\ e)@b & = & (\oddFrom f\ (f^{2^k-1}x)\ e)@\nil \\
& = & f^{2^k-1}x \\
& = & f^{0+2^k-1}x \\
& = & f^{2^k\ord{\nil}+2^k-1}x
\end{array}
\end{displaymath}

If $b = \cons \true d$,
\begin{displaymath}
\begin{array}{rcl}
(\oddFrom f\ (f^{2^k-1}x)\ e)@(\cons \true d)& = & \\
(\oddFrom f\ (f (\head e))\ (\evens e))@d& = & \\
(\oddFrom f\ (f (f^{2^{k+1}-2}e))\ (\evens e))@d& = & \\
(\oddFrom f\ (f^{2^{k+1}-1}e)\ (\evens e))@d&  & 
\end{array}
\end{displaymath}

Now, for all $g$ such that $\ord{g} < \ord{d}$, $\ord{g} \leq \ord{d}-1$, so 
\begin{displaymath}
\ord{\cons \false g} = 2 + 2\ord{g} \leq 2 + 2(\ord{d} -1) = 2\ord{d} < 1+2\ord{d} = \ord{\cons \true d}
\end{displaymath}

So, $\ord{\cons \false g} < \ord{b}$, and 
\begin{displaymath}
\begin{array}{rcccl}
(\evens e)@g & = & e@(\cons \false g) & = & f^{2^k\ord{\cons \false g}+2^{k+1}-2}x \\
& & & = & f^{2^k(2+2\ord{g})+2^{k+1}-2}x \\
& & & = & f^{2^{k+1} + 2^{k+1}\ord{g}+2^{k+1}-2}x \\
& & & = & f^{2^{k+1}\ord{g}+2^{k+2}-2}x 
\end{array}
\end{displaymath}

Since this holds for all $g < d$ we can invoke the induction hypothesis with $k := k+1$ and $e := \evens e$.
This implies that

\begin{displaymath}
\begin{array}{rcl}
(\oddFrom f\ (f^{2^k-1}x)\ e)@b & = & (\oddFrom f\ (f^{2^{k+1}-1}x)\ (\evens e))@d \\
& = & f^{2^{k+1}d+2^{k+1}-1} x \\
& = & f^{2^k2d+2^k+2^k-1} x \\
& = & f^{2^k(2d+1)+2^k-1} x \\
& = & f^{2^k\ord{\cons \true d}+2^k-1} x \\
& = & f^{2^k\ord{b}+2^k-1} x
\end{array}
\end{displaymath}

If $b = \cons \false d$,

\begin{displaymath}
\begin{array}{rcl}
(\oddFrom f\ (f^{2^k-1}x)\ e)@(\cons \false d)& = & (\fmap f\ (\odds e))@d \\
& = & f ((\odds e)@d)
\end{array}
\end{displaymath}

Since
$\ord{\cons \true d} = 1+2\ord{d} < 2+2\ord{d} = \ord{\cons \true d}$,

\begin{displaymath}
\begin{array}{rcccl}
f((\odds e)@d) & = & f(e@(\cons \true d)) & = & f(f^{2^k\ord{\cons \true d}+2^{k+1}-2}x) \\
& & & = & f^{1+2^k(1+2\ord{d})+2^{k+1}-2}x \\
& & & = & f^{2^k+2^{k+1}\ord{d}+2^{k+1}-1}x \\
& & & = & f^{2^k(2+2\ord{d})+2^k-1}x \\
& & & = & f^{2^k\ord{\cons \false d}+2^k-1}x \\
& & & = & f^{2^k\ord{b}+2^k-1}x
\end{array}
\end{displaymath}

\end{proof}

\begin{corollary}\label{oddFromCorollary}
For any $f, b, x, e$
if
\begin{displaymath}
\forall j . \ord{j} < \ord{b} \implies e@j = f^{\ord{\cons \false j}}x
\end{displaymath}
then
\begin{displaymath}
(\oddFrom f\ (f x)\ e)@b = f^{\ord{\cons \true b}}x
\end{displaymath}
\end{corollary}
\qed

\begin{theorem}\label{iterateCorrect}
\begin{displaymath}
\forall f\ x\ b, (\iter f\ x)@b = f^\ord{b} x
\end{displaymath}
\end{theorem}
\begin{proof}

We proceed by well-founded induction on $\ord{b}$.
If $b = \nil$, the equality is trivial.
Otherwise, let $b = \cons a\ d$.
There is a lemma that is needed whether $a$ is $\true$ or $\false$

\begin{lemma}\label{iterateSublemma}
\begin{displaymath}
(\oddFrom f\ (f x)\ \ev)@d = f^{2\ord{d}+1}x
\end{displaymath}
\end{lemma}
\begin{proof}
This is a direct result of the corollary as long as 
\begin{displaymath}
\forall j, \ord{j} < \ord{d} \implies \ev @j = f^{2\ord{j}+2}
\end{displaymath}

Since $\ord{j} < \ord{d}$, $2\ord{j}+2 < 2\ord{d}+2 <= \ord{b}$, so we can apply the induction hypothesis to $2\ord{j}+2 = \ord{\cons \false\ j}$:

\begin{displaymath}
\begin{array}{rcl}
(\iter f\ x)@(\cons \false\ j) & = & f^(\ord{\cons \false\ j}) x \\
\ev @j & = & f^{2\ord{j}+2}
\end{array}
\end{displaymath}

\end{proof}

Now, when $a = \true$, $(\iter f\ x)@(\cons \true\ d) = (\oddFrom f\ (f x)\ \ev)@d$.
By lemma~\ref{iterateSublemma}, this is just $f^{2\ord{d}+1}x = f^{\ord{b}}x$

When $a = \false$, 

\begin{displaymath}
\begin{array}{rcl}
(\iter f\ x)@(\cons \false\ d) & = & (\fmap f\ \od)@d \\
& = & f(\od @d) \\
& = & f((\oddFrom f\ (f x)\ \ev)@d) \\
& = & f(f^{2\ord{d}+1}x) \\
\end{array}
\end{displaymath}

With the last step justified by lemma~\ref{iterateSublemma}. This is just $f^{2\ord{d}+2}x = f^{\ord{\cons \false\ d}}x = f^{\ord{b}}x$.

\end{proof}

\bibliographystyle{plain}
\bibliography{braun}

\end{document}
